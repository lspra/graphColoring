\documentclass[diplomskirad]{fer}
% Dodaj opciju upload za generiranje konačne verzije koja se učitava na FERWeb
% Add the option upload to generate the final version which is uploaded to FERWeb

%--- PODACI O RADU / THESIS INFORMATION ----------------------------------------

\newtheorem{definicija}{Definicija}
\newtheorem{theorem}{Teorem}[chapter]
% Naslov na engleskom jeziku / Title in English
\title{Semidefinite programming relaxation for graph coloring and maximum clique number}

% Naslov na hrvatskom jeziku / Title in Croatian
\naslov{Relaksacije semidefinitnog programiranja za probleme bojanja grafova i
maksimalne klike}

% Broj rada / Thesis number
\brojrada{819}

% Autor / Author
\author{Lana Šprajc}

% Mentor 
\mentor{prof. dr. sc. Josipa Pina Milišić}

% Datum rada na engleskom jeziku / Date in English
\date{June, 2025}

% Datum rada na hrvatskom jeziku / Date in Croatian
\datum{lipanj, 2025.}


%-------------------------------------------------------------------------------


\begin{document}


% Naslovnica se automatski generira / Titlepage is automatically generated
\maketitle


%--- ZADATAK / THESIS ASSIGNMENT -----------------------------------------------

% Zadatak se ubacuje iz vanjske datoteke / Thesis assignment is included from external file
% Upiši ime PDF datoteke preuzete s FERWeb-a / Enter the filename of the PDF downloaded from FERWeb
\zadatak{hr_0036532785_94.pdf}


%--- ZAHVALE / ACKNOWLEDGMENT --------------------------------------------------

\begin{zahvale}
  % Ovdje upišite zahvale / Write in the acknowledgment
  Hvala Branimiru na podsjetnicima da trebam pisati diplomski...
\end{zahvale}


% Odovud započinje numeriranje stranica / Page numbering starts from here
\mainmatter


% Sadržaj se automatski generira / Table of contents is automatically generated
\tableofcontents


%--- UVOD / INTRODUCTION -------------------------------------------------------
\chapter{Uvod}
\label{pog:uvod}

U teoriji grafova, problemi bojanja grafa i traženja maksimalne klike poznati su NP-teški problemi s brojnim primjenama. Slijedi opis
ovih problema. Bojanje grafa svakom vrhu grafa pridružuje jednu boju na način da su svaka dva susjedna vrha različitih boja. Poznate primjene ovog
problema su alociranju registara, raspoređivanju zadataka\dots  %TODO izvor za primjenu %TODO ovdje ili poslije treba spomenuti 3 color theorem

Problem traženja maksimalne klike je problem pronalaska potpuno povezanog podgrafa. Problem ima česte primjene u kemiji i bioinformatici,
primjerice za pronalaženje sličnih struktura u molekulama. %TODO izvor za primjenu

Za oba ovo problema često je dovoljno pronaći rješenje koje je blizu optimalnom. Semidefinitno programiranja daje dobre aproksimacije rješenja problema
koje se mogu izračunati u polinomnoj složenosti.

%TODO
Rad je podijeljen u X poglavlja. U svakom je opisano ovo.

%-------------------------------------------------------------------------------
\chapter{Opis problema}
\label{opis_problema}

Graf se sastoji od $n$ vrhova i $m$ bridova. Pišemo $G = (V, E)$. Stupanj vrha je definiran kao broj vrhova s kojima je taj vrh susjedan. $\delta$ je
oznaka za najveći stupanj vrha.

\section{Bojanje grafova}
Bojanje grafa je postupak pridruživanja boje svakom vrhu grafa tako da dva susjedna vrha nemaju jednaku boju. Graf je $k$-obojiv ako postoji
bojanje grafa s $k$ ili manje boja. Kromatski broj $\chi(G)$ grafa je najmanji broj boja potreban za obojati graf. U većini primjena,
dovoljno je obojati graf s dovoljno malim $k$ boja, $k > \chi(G)$.

\subsection{$\delta + 1$ bojanje grafa}
$\delta + 1$ je jednostavan algoritan bojanja grafa koji oboja graf s najviše $\delta + 1$ bojom. Algoritam je složenosti $O(n^2)$.
Neka je $\{1, \dots, \delta + 1\}$ skup boja. Svakom vrhu se pridružuje najmanji broj kojim nije obojan ni jedan od njegovih susjeda. Uvijek će postojati
barem jedna takva boja jer svaki vrh ima najviše $\delta$ susjeda, a postoji $\delta + 1$ boja. Problem ovog algoritma je što je u većini slučajeva 
kromatski broj grafa puno manji od $\delta + 1$.

\section{Traženje maksimalne klike}
Klika u grafu je potpuno povezani podgraf. Maksimalna klika (engl. \textit{maximum clique}) je najveći takav podgraf. Broj vrhova u maksimalnoj kliki označavamo s $\omega(G)$.
Ovaj broj predstavlja donju granicu kromatskog broja grafa, $\omega(G) \leq \chi(G)$.

Lokalno maksimalna klika (engl. \textit{maximal clique, inclusion-maximal}) je klika koja nije podgraf veće klike. Takvoj kliki nije moguće dodati ni jedan
vrh grafa koji nije dio klike, na način da dobiveni podgraf ostaje klika.

\subsection{Pohlepno traženje lokalno maksimalne klike}
Jednostavan algoritam za traženje lokalno maksimalne klike započinje s proizvoljnim vrhom $S = \{v\}$. Algoritam iterira po preostalim vrhovima i provjerava ostaje li
podgraf $S$ klika ako mu pridružimo odabrani vrh $w$. Ako ostaje, vrh pridružimo lokalno maksimalnoj kliki, a u suprotnom ga odbacujemo.

\chapter{Semidefinitno programiranje}
\label{pog:semidefinitno_programiranje}
Semidefinitno programiranje (krat. \textit{SDP}) klasa je optimizacijskih problema.
Standardni oblik problema je:

\begin{equation}
\begin{split}
  & min \langle CX \rangle \\
  t. d. & \langle A_iX \rangle = b_i \\
        & X = X^T \succeq 0
\end{split}
\end{equation}
pri čemu $ \langle A \rangle = \sum_{i}^{n} a_{i,i} $
označava trag kvadratne matrice. $X \succeq 0$ označava da je matrica pozitivno semidefinitna.

Dualni problem zadan je s jednadžbama: 
\begin{equation}
\begin{split}
  & min \sum_{i}^{m} b_iy_i \\
  t.d. & \sum_{i}^{m} A_iy_i + Z = C \\
      & Z \succeq 0 \\
\end{split}
\end{equation}

%TODO to bi trebalo iti neka primjena SDP

\section{Teorija dualnosti}
U optimizacijskim problemima, razmak dualnosti (engl. \textit{duality gap}) je razlika između optimuma primarnog $p^*$ i dualnog $d^*$ problema, $p^* - d^*$.
Kažemo da vrijedi slaba dualnost ako je razmak dualnosti veći ili jednak od 0. Jaka dualnost vrijedi ako su optimum primarnog i dualnog problema jednaki, to jest
ako je razmak dualnosti jednak 0. % TODO layon mi je citat

Za SDP je razmak dualnosti jednak:
\begin{equation}
  \begin{split}
    \langle CX \rangle - b^Ty &= \langle(\sum_{i=1}^{m}A_iy_i + Z)X \rangle - b^Ty \\
    &= \langle ZX \rangle + \sum_{i=1}^{m}(\langle A_iX \rangle - b_i) y_i \\
    &= \langle ZX \rangle \geq 0
  \end{split}
\end{equation}

Iz ovoga vidimo da za SDP vrijedi slaba dualnost, ali ne nužno i jaka. Prema Slaterovom uvjetu, jaka dualnost je zadovoljena ako postoji rješenje primarnog problema u kojem je
zadovoljen uvjet $X \succ 0$ i rješenje dualnog problema takvo da vrijedi $Z \succ 0$. %TODO citat za Slaterov uvjet

\section{Metode rješavanja SDP}
Postoje razne metode rješavanja SDP-a, primjerice elipsoidna metoda i metoda unutarnje točke. Ove metode numerički pronalaze rješenje, obično u polinomnoj složenosti.

\subsection{Metoda unutarnje točke}
Kao što ime kaže, metoda unutarnje točke započinje s nekom točkom unutar prostora koji zadovoljava uvjete semidefinitnog programa i priližava se optimumu. 
%TODO ovo treba dosta bolje objasniti ali neda mi se skuziti 

\chapter{Lovászov broj}
\label{pog:lovaszov_broj}
Lovászov broj (Lovászova theta funkcija) rješenje je sljedećeg semidefinitnog programa:
\begin{equation}
  \begin{split}
    & \theta = \max \langle JX \rangle \\
    t.d. & \langle X \rangle = 1 \\
         & x_{i,j} = 0, \forall (i,j) \in E(G) \\
         & X=X^T \succeq 0 
  \end{split}
\end{equation}

Ovdje J predstavlja $n \times n$ matricu ispunjenu jedinicama.


Za Lovászov broj, $ \theta(G) $ vrijedi da je veći od ili jednak broju klike, a manji od ili jednak kromatskom broju grafa, 
$\omega(G) \leq \theta(G) \leq \chi(G)$. \cite{1055985}

Ovaj program nije teško pretvoriti u standardnu formu. Potrebno je postaviti $C = -J$, $A_1 = I$, $b_1 = 1$,
$A_i = E_{j,k}, \forall (j,k) \in E(G)$, $b_i = 0, i \in \{2, \dots, m\}$. Ovdje $E_{i,k}$ predstavlja matricu koja sadrži 0 na
svim mjestima osim na $e_{i,j}$ i $e_{j, i}$ gdje sadrži 1.


\chapter{Primjene SDP-a u bojanju grafova}
\label{pog:primjene_SDP-a_u_bojanju_grafova}

U formulaciji bojanja grafova pomoću SDP-a, vrhovima se pridružuju jedinični vektoru. Vektori pridruženi susjednim vrhovima moraju
biti udaljeni jedan od drugog.

\begin{definicija}
  Za graf G=(V, E) s n vrhova i m bridova i realni broj $k \geq 1$, vektorsko k-bojanje je pridruživanje jediničnih
  vektora $v_i \in \Re^n$ svakom vrhu $i$ grafa G takvih da za svaka dva susjedna vrha $v_{i,j} \in E$ vrijedi:
  \begin{equation}
    v_i \cdot v_j \leq - \frac{1}{k-1}
  \end{equation}
\end{definicija} \cite{karger1998approximategraphcoloringsemidefinite}

\begin{definicija}
  Za graf G=(V, E) s n vrhova i m bridova i realni broj $k \geq 1$, matrično k-bojanje je određivanje simetrične pozitivno semidefinitne matrice M,
  takve da je $m_{i,i} = 1, i \in \{1, \dots, n\}$ i $m_{i,j} = m_{j,i} \leq - \frac{1}{k-1}, (i, j) \in E$.
\end{definicija}

Matrično k-bojanje i vektorsko k-bojanje su ekvivalentni. Ako je poznato vektorsko k-bojanje, matricu M određuje se tako da se
element $m_{i,j}$ postavi na skalarni produkt vektora $v_i$ i $v_j$. Ako je poznato matrično k-bojanje, matricu $M$ moguće je faktorizirati
pomoću dekompozicije svojstvenim vrijednosti (engl. \textit{eigendecomposition}) tako da vrijedi $M = UU^T$. Sada su stupci matrice $U$, jedinični
vektori koji čine vektorsko k-bojanje grafa. \cite{karger1998approximategraphcoloringsemidefinite}

%TODO ovo je pol teorema iz KMS, cini mi se dovoljno al treba prouciti
\section{Povezanost vektorskog bojanja grafa i bojanja grafa}
\begin{theorem}\label{teorem1}
  Za $k \leq n$, postoji k jediničnih vektora $v_i \in \Re^n$ takvih da je $v_i \cdot v_j = -\frac{1}{k-1}, \forall i \neq j$ 

  \textbf{Dokaz} (iz \cite{karger1998approximategraphcoloringsemidefinite}) Dovoljno je dokazati da ovo vrijedi za k = n, ako je k manji od n
  $v_{i}[j]$ se postavlja na 0 za sve j > k.

  Konstruirani su vektori $v_1, \dots, v_k$ na sljedeći način:

  \begin{equation}
    v_i[j] = \begin{cases} -\sqrt{\frac{1}{k(k-1)}} & j \neq i \\ \sqrt{\frac{k-1}{k}} & j = i \end{cases}
  \end{equation}
\end{theorem}

\begin{theorem}
  Svaki k-obojiv graf je i vektorski k-obojiv.

  \textbf{Dokaz} (iz \cite{karger1998approximategraphcoloringsemidefinite}) Graf je uvijek obojiv s $k \leq n$ boja jer je u najgorem slučaju
  svaki vrh obojan u različitu boju. Svakom vrhu obojanom s $i$-tom bojom, pridružuje se vektor $v_i$ iz \ref{teorem1}.
\end{theorem}

Iz ovog teorema vidljivo je da je minimalni k za koji je graf vektorski k-obojiv donja granica za kromatski broj grafa.

\section{Formulacija SDP}
Slijedi SDP formulacija vektorskog bojanja grafa koja minimizira k za koji je graf vektorski k-obojiv.

\begin{equation}
  \begin{split}
    & \min \alpha \\
    t.d. & x_{ii} = 1 \\
         & x_{ij} \leq \alpha, \forall (i,j) \in E(G) \\
         & X=X^T \succeq 0 
  \end{split}
\end{equation}

Optimalni $\alpha = -\frac{1}{k-1}$. Kako bi problem bio u standardnoj formi, u matrici $X$ uvodi se novi element na dijagonalu tako da
$x_{n+1,n+1} = \alpha$. Dodatno, uvode se \textit{slack} varijable za uvjete povezane s bridovima. Konačna formulacija problema u standarnoj formi je: 


\begin{equation}
  \begin{split}
    & \min \langle E_{n+1,n+1}X \rangle \\
    t.d. & \langle E_{ii}X \rangle = 1 \\
         & \langle E_{ij}X \rangle - 2 \langle E_{n+1,n+1}X \rangle + \langle E_{n+1+k,n+1+k}X \rangle = 0, \forall (i,j) \in E(G), k \in \{1,\dots,m\} \\
         & X=X^T \succeq 0 
  \end{split}
\end{equation}

\section{Pretvorba vektorskog bojanja u bojanje grafa}
Rješavanjem SDP-a, dobiva se vektorsko bojanje grafa. Ovo je potrebno pretvoriti u bojanje grafa. Postoji nekoliko algoritma za ovaj problem. Neki od njih su metoda odsijecanja ravninama 
(engl. \textit{cutting plane method}) i zaokruživanje vektorskim projekcijama.

\subsection{Semibojanja}
\begin{definicija}
  k-semibojanje (engl. \textit{k-semicoloring}) grafa G je pridruživanje k boja barem polovici vrhova grafa G tako da dva susjedna vrha nisu jednake boje. 
\end{definicija}

\subsection{Metoda odsijecanja ravninama}
Motivacija iza korištenja algoritma je činjenica da su vektori koji predstavljaju susjedne vrhove udaljeni u prostoru, to jest velik je kut između njih. U algoritmu 

\subsection{Zaokruživanje vektorskim projekcijama}

\chapter{Primjene SDP-a u traženju maksimalne klike}
\label{pog:primjene_SDP-a_u_traženju_maksimalne_klike}

\chapter{Programska ostvarenja}
\label{pog:programska_ostvarenja}
\section{Bojanje grafova}
\section{Traženje maksimalne klike}


%-------------------------------------------------------------------------------
\chapter{Rezultati i rasprava}
\label{pog:rezultati_i_rasprava}

\chapter{Upute za prevođenje i pokretanje}
\label{pog:upute_za_prevođenje_i_pokretanje}

%--- ZAKLJUČAK / CONCLUSION ----------------------------------------------------
\chapter{Zaključak}
\label{pog:zakljucak}


%--- LITERATURA / REFERENCES ---------------------------------------------------

% Literatura se automatski generira iz zadane .bib datoteke / References are automatically generated from the supplied .bib file
% Upiši ime BibTeX datoteke bez .bib nastavka / Enter the name of the BibTeX file without .bib extension
\bibliography{literatura}



%--- SAŽETAK / ABSTRACT --------------------------------------------------------

% Sažetak na hrvatskom
\begin{sazetak}
  Unesite sažetak na hrvatskom.

\end{sazetak}

\begin{kljucnerijeci}
  prva ključna riječ; druga ključna riječ; treća ključna riječ
\end{kljucnerijeci}


% Abstract in English
\begin{abstract}
  Enter the abstract in English.
  
\end{abstract}

\begin{keywords}
  the first keyword; the second keyword; the third keyword
\end{keywords}


%--- PRIVITCI / APPENDIX -------------------------------------------------------

% Sva poglavlja koja slijede će biti označena slovom i riječi privitak / All following chapters will be denoted with an appendix and a letter
\backmatter

\end{document}
